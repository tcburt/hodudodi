

\begin{frame}
  \frametitle{Further Reading}
  \begin{itemize}
  \item Quantum Computing Reports, Tools~\cite{qcomprpt-tools}
  \item Quantum Programming (Wikipedia)~\cite{wikipedia-qcomp}
  \item Quantum Computing: An Overview Across the System Stack~\cite{arxiv:1905.07240}
  \item Full-Stack, Real-System Quantum Computer Studies~\cite{arxiv:1905:11349} 
    \begin{itemize}
    \item ``the purpose of our work is to build insights in how architecture and compiler design choices can best support the different technologies''
    \end{itemize}
  \item Quantum Networks for Elementary Arithmetic Operations~\cite{arXiv:9511018}
  \item Automated optimization of large quantum circuits with continuous parameters~\cite{npj-s41534-018-0072-4}
  \end{itemize}
\end{frame}


\begin{frame}
  \frametitle{Abstract}
  In analogy with a classical computing stack, aspects of quantum information processing are explored from physical underpinnings of devices to application programming. Particular emphasis is given to the subsystems between these extremes regarding challenges to realize them such as scaling and error correction. Potential applications that could benefit from quantum processing and the skills of computer engineering are discussed, especially those that might be considerably more near-term than a universal quantum computer. As Tank said in `Matrix'~\cite{movie-Matrix}, ``It’s a very exciting time. We got a lot to do. We gotta get to it.''
\end{frame}


\begin{frame}[allowframebreaks]{Bibliography}
  \begin{thebibliography}{99}
  \bibitem{hist-comp-SSEM} %
    \emph{SSEM (Small Scale Experimental Machine)}, %
    \url{https://history-computer.com/ModernComputer/Electronic/SSEM.html}

  \bibitem{acm-qcomp-blueprint} %
    \emph{A Blueprint For Building a Quantum Computer}, %
    Rodney Van Meter and Clare Horsman, %
    Communications of the ACM, Vol. 56 No. 10, Pages 84-93, (2013) %
    \url{https://m-cacm.acm.org/magazines/2013/10/168172-a-blueprint-for-building-a-quantum-computer/fulltext}

  \bibitem{chm-ENIAC} %
    \emph{ENIAC (Birth of the Computer)}, %
    Computer History Museum, %
    \url{https://www.computerhistory.org/revolution/birth-of-the-computer/4/78}

  \bibitem{Crystal-Fire} %
    \emph{Crystal Fire: The Invention of the Transistor and the Birth of the Information Age}, %
    Mitchael Rioradan and Lillian Hoddeson, %
    W.W. Norton and Company, (1997)
    
  \bibitem{QI-History-Rhymes} %
    \emph{History Does Not Repeat Itself, But It Rhymes}, %
    Quote Investigator, %
    \url{https://quoteinvestigator.com/2014/01/12/history-rhymes}

    
  \bibitem{Sheffer} %
    \emph{A set of five independent postulates for Boolean algebras, with application to logical constants}, %
    Henry Maurice Sheffer, %
    Trans. Amer. Math. Soc. 14 481-488, (1913)
  \bibitem{electronic-tutorials} %
    \emph{Universal Logic Gates}, %
    \url{https://www.electronics-tutorials.ws/logic/universal-gates.html}
    
  \bibitem{QCQI-Nielsen} %
    \emph{Quantum Computation and Quantum Information},
    Michael A. Nielsen and Isaac L. Chuang, %
    Cambridge University Press, %
    2000

  \bibitem{IBMQ-blog} %
    \emph{The future is quantum}, %
    IBM Research Editorial Staff, %
    IBM Research Blog, (2017), %
    \url{https://www.ibm.com/blogs/research/2017/11/the-future-is-quantum}

  \bibitem{Xanadu-Hardware} %
    Xanadu Hardware, %
    date retrieved 2019-09-29, 
    \url{https://www.xanadu.ai/hardware/}

  \bibitem{JQI-15M-NSF-news} %
    \emph{JQI scientists Monroe and Gorshkov are part of a new, \$15 million
      NSF quantum computing project}, %
    Joint Quantum Institute, University of Maryland, (2018), %
    \url{https://jqi.umd.edu/news/jqi-scientists-monroe-and-gorshkov-are-part-new-15-million-nsf-quantum-computing-project}

  \bibitem{CHM-Transistor-article} %
    \emph{Inventing the Transistor}, %
    Computer History Museum, %
    \url{https://www.computerhistory.org/revolution/digital-logic/12/273}

  \bibitem{PW-Shor-factor-15} %
    \emph{Shor’s algorithm is implemented using five trapped ions}, %
    Hamish Johnston, Physics World, (2016), %
    \url{https://physicsworld.com/a/shors-algorithm-is-implemented-using-five-trapped-ions/}

  \bibitem{QITheory-Ingarden} %
    \emph{Quantum Information Theory}, %
    Roman Ingarden, %
    Reports on Mathematical Physics, %
    Volume 10, Issue 1, Pages 43-72, (1976)

  \bibitem{NIST-QComp} %
    \emph{The History and Future of Quantum Information}, %
    Ben P. Stein, National Institute of Standards and Technology, %
    \url{https://www.nist.gov/history-and-future-quantum-information}

  \bibitem{Wikipedia-Timeline-QComp} %
    \emph{Timeline of Quantum Computing}, %
    Wikipedia, date accessed 2019-10-01, %
    \url{https://en.wikipedia.org/wiki/Timeline_of_quantum_computing}

  \bibitem{QAlg-zoo} %
    \emph{Quantum Algorithm Zoo}, %
    \url{https://quantumalgorithmzoo.org}

  \bibitem{NIST-Q-Logic-Gates} %
    \emph{Quantum Logic Gates}, %
    NIST, %
    \url{https://www.nist.gov/topics/physics/introduction-new-quantum-revolution/quantum-logic-gates}

  \bibitem{SC-QComp-stack} %
    \emph{Building logical qubits in a superconducting quantum computing
      system}, %
    Jay M. Gambetta, Jerry M. Chow, and Matthias Steffen, %
    npj Quantum Information volume 3, Article number: 2 (2017), %
    \url{https://www.nature.com/articles/s41534-016-0004-0}

  \bibitem{VOQC} %
    \emph{A Verified Optimizer for Quantum Circuits (draft)}, %
    Kesha Hietala, Robert Rand, Shih-Han Hung, Xiaodi Wu, and Michael
    Hicks, %
    \url{http://www.cs.umd.edu/~rrand/voqc_draft.pdf}

  \bibitem{Lost-history-transistor} %
    \emph{The Lost History of the Transistor}, %
    Michael Riordan, IEEE Spectrum, (2004), %
    \url{https://spectrum.ieee.org/tech-history/silicon-revolution/the-lost-history-of-the-transistor}

  \bibitem{NRL-nanoelectronics}, %
    \emph{Nanoelectronics}, %
    U.S. Naval Research Laboratory Electronics Science and Technology
    Division, %
    \url{https://www.nrl.navy.mil/estd/research-highlights/nanoelectronics}
    
  \bibitem{PhysOrg-3nm-channel-transistor}, %
    \emph{Success in operation of transistor with channel length of 3
      nm}, %
    Advanced Industrial Science and Technology, (2013), %
    \url{https://phys.org/news/2013-02-success-transistor-channel-length-nm.html}

  \bibitem{NIST-Single-electron-transport}, %
    \emph{NIST-on-a-Chip: Quantum-Based Electrical Standards - Current}, %
    NIST Physical Measurement Laboratory, %
    \url{https://www.nist.gov/pml/nist-chip-quantum-based-electrical-standards-current}

    
  % ----------------------------------------------------------
  % Further reading
  \bibitem{qcomprpt-tools} %
    \emph{Tools (for working with quantum computation)}, %
    \url{https://quantumcomputingreport.com/resources/tools/}
  \bibitem{wikipedia-qcomp} %
    \emph{Quantum computing}, Wikipedia, date retrieved 28 Sep 2019 %
 \url{https://en.wikipedia.org/w/index.php?title=Quantum_computing&oldid=918292451}
  \bibitem{arxiv:1905.07240}%
    \emph{Quantum Computing: An Overview Across the System Stack}, %
    Salonik Resch and Ulya R. Karpuzcu, (2019) %
    \url{https://arxiv.org/abs/1905.07240}
  \bibitem{arxiv:1905:11349} %
    \emph{Full-Stack, Real-System Quantum Computer Studies: Architectural Comparisons and Design Insights}, %
    Prakash Murali, Norbert Matthias Linke, Margaret Martonosi, Ali Javadi Abhari, Nhung Hong Nguyen, and Cinthia Huerta Alderete, %
   (2019) \url{https://arxiv.org/abs/1905.11349}
  \bibitem{arXiv:9511018} %
    \emph{Quantum Networks for Elementary Arithmetic Operations}, %
    V. Vedral, A. Barenco and A. Ekert, (1995) %
    \url{https://arxiv.org/abs/quant-ph/9511018}
  \bibitem{npj-s41534-018-0072-4} %
    \emph{Automated optimization of large quantum circuits with continuous
      parameters}, %
    Yunseong Nam, Neil J. Ross, Yuan Su, Andrew M. Childs and Dmitri Maslov, %
    npj Quantum Information volume 4, Article number: 23 (2018) %
    \url{https://www.nature.com/articles/s41534-018-0072-4}
  \bibitem{movie-Matrix} %
    \emph{Matrix} movie written and directed by The Wachowskis, (1999)
  \end{thebibliography}
\end{frame}
