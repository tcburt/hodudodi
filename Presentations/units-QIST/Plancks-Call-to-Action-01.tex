%-req-%\usetikzlibrary{shadows}

%-req-%\newcommand{\AxisRotator}[1][rotate=0]{%
%-req-%    \tikz \draw[x=.5em,y=1.25em,line width=.2ex,-stealth,#1] (0,0) arc (-150:150:1 and 1);%
%-req-%  }

%-req-%\tikzset{shaded/.style args={#1:#2:#3 @ #4}{
%-req-%  left color=#1, right color=#3, middle color=#2, shading angle=#4
%-req-%}}
%-req-%\tikzset{pics/.cd, clock/.style args={#1:#2:#3}{code={
%-req-%%\tikzset{x=1ex, y=1ex, every path/.style={line cap=round}} 
%-req-%\tikzset{x=0.5ex, y=0.5ex, every path/.style={line cap=round}} 
%-req-%\shade [shaded=black!75:black!50:black!25 @ 225] circle [radius=13];
%-req-%\shade [shaded=black!75:black!50:black!25 @ 45]  circle [radius=12.5];
%-req-%\fill [black!90] circle [radius=12];
%-req-%\foreach \i [evaluate={\j=90-\i*30; \k=mod(\i,3)==0;\m=int(\i*5);}] in {1,...,12}
%-req-%  \draw [white, line width=\k ? .2ex : .1ex] (\j:11.5) -- (\j:11-\k) 
%-req-%    (\j:10) node [anchor=\j, text=black!20, font=\tiny\sffamily]  {\expandafter\uppercase\expandafter{\romannumeral\i}};
%-req-%\shade [inner color=white, outer color=black, opacity=0.25] circle [radius=12];
%-req-%\fill [gray!50, rotate=90-#1*30-#2/2-#3/120, 
%-req-%  rounded corners=.25ex, drop shadow={fill=black, opacity=0.5}]
%-req-%  (-3/2,3/4) -- (-3/2,-3/4) -- (7,0) -- cycle;
%-req-%\fill  [gray!50, rotate=90-#2*6-#3/60, 
%-req-%  rounded corners=0.25ex, drop shadow={fill=black, opacity=0.5}]
%-req-%  (-3/2,3/4) -- (-3/2,-3/4) -- (11,0) -- cycle;
%-req-%\fill [red!75!black, drop shadow={fill=black, opacity=0.5}, rotate=90-#3*6] (0,-.05ex) rectangle (11,.05ex);
%-req-%\shade [shaded=black!50:black!25:black!10 @ 225] circle [radius=1];
%-req-%\shade [shaded=black!50:black!25:black!10 @ 45] circle [radius=3/4];
%-req-%\fill [white, opacity=1/16] (0:11) arc (0:180:11) 
%-req-%.. controls ++(-45:10) and ++(135:10) .. cycle;
%-req-%}}}
  
\begin{frame}
  \frametitle{Planck's Call to Action}

  The difference between classical physics and quantum
  physics is Planck's constant $\hbar$ (technically, $i\hbar$). %
  \mbox{}
  \vfill

  \begin{tikzpicture}
    \node (hbarapprox) at (-2.0,1.0) {\Huge $\hbar \approx$};    
    %foreach \c [count=\i from 0] in {white,gray,red!75!black,blue!25, purple,orange}{

%\tikzset{xshift={mod(\i,2)*3cm}, yshift=-floor(\i/2)*3cm}
\colorlet{cup}{blue!25}

% Saucer
\begin{scope}[shift={(0,-1-1/16)}]
    \fill [black!87.5, path fading=fade out] 
      (0,-2/8) ellipse [x radius=6/4, y radius=3/4];
    \fill [cup, postaction={left color=black, right color=white, opacity=1/3}] 
      (0,0) ++(180:5/4) arc (180:360:5/4 and 5/8+1/16);
    \fill [cup, postaction={left color=black!50, right color=white, opacity=1/3}] 
      (0,0) ellipse [x radius=5/4, y radius=5/8];
    \fill [cup, postaction={left color=white, right color=black, opacity=1/3}]
      (0,1/16) ellipse [x radius=5/4/2, y radius=5/8/2];
    \fill [cup, postaction={left color=black, right color=white, opacity=1/3}] 
      (0,0) ellipse [x radius=5/4/2-1/16, y radius=5/8/2-1/16];
\end{scope} 

% Handle
\begin{scope}[shift=(10:7/8), rotate=-30, yslant=1/2, xslant=-1/8]
  \fill [cup, postaction={top color=black, bottom color=white, opacity=1/3}] 
    (0,0) arc (130:-100:3/8 and 1/2) -- ++(0,1/4) arc (-100:130:1/8 and 1/4) -- cycle;
  \fill [cup, postaction={top color=white, bottom color=black, opacity=1/3}] 
    (0,0) arc (130:-100:3/8 and 1/2) -- ++(0,1/32) arc (-100:130:1/4 and 1/3) -- cycle;
\end{scope}

% Cup
\fill [cup!25!black, path fading=fade out] 
    (0,-1-1/16) ellipse [x radius=3/4, y radius=1/3];
\fill [cup, postaction={left color=black, right color=white, opacity=1/3/2},
  postaction={bottom color=black, top color=white, opacity=1/3/2}] 
    (-1,0) arc (180:360:1 and 5/4);
\fill [cup, postaction={left color=white, right color=black, opacity=1/3}] 
  (0,0) ellipse [x radius=1, y radius=1/2];
\fill [cup, postaction={left color=black, right color=white, opacity=1/3/2},
  postaction={bottom color=black, top color=white, opacity=1/3/2}] 
    (0,0) ellipse [x radius=1-1/16, y radius=1/2-1/16];

% Coffee
\begin{scope}
\clip ellipse [x radius=1-1/16, y radius=1/2-1/16];
\fill [brown!25!black] 
  (0,-1/4) ellipse [x radius=3/4, y radius=3/8];
\fill [brown!50!black, path fading=fade out] 
  (0,-1/4) ellipse [x radius=3/4, y radius=3/8];
\end{scope}
%}
    \draw[->, >=stealth, gray] (0,-0.2) -- ++(0,2);
    \draw[ultra thick] (0,1.5) node {\AxisRotator[rotate=-90]};
    \node [right] (rot) at (0.5, 1.5) {rotating a cup a billionth of a turn
      \ldots};
    \node[right] (clock) at (2.0,1.0) {\ldots over the lifetime of the
      universe!};
    
    \pic at (5.5,-0.5) {clock={6:36:34}};
    \node[below] (hbar) at ((5.5,-1.5) {$\hbar \equiv \SI{1.054571817e-34}{\joule\second}$};
  \end{tikzpicture}
\end{frame}
