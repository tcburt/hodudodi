\section{Work and Energy}

Work is the result of a motive cause acting over a path
\begin{align*}
  dW &= \vec{F}\cdot d\vec{s}   & dW &= \vec{\tau} \cdot d\theta \hat{q}\\
  dW &= \vec{F}\cdot \vec{v}dt   & dW &= \vec{\tau} \cdot \vec{\omega} dt.
\end{align*}
Kinetic energy expressions for translational and rotational motion are
\begin{align*}
  K &= \tfrac{1}{2}mv^2   & K &= \tfrac{1}{2}I\omega^2.
\end{align*}
Two conservative potential energies of interest are gravitational and elastic
\begin{align*}
  U_{\text{grav}} &= mgh\\
  U_{\text{el}} &= \tfrac{1}{2} k x^2.
\end{align*}
Conservative potentials yield forces via the spatial gradient (derivative)
\begin{equation*}
  \vec{F} = -\nabla U_{\text{cons}}.
\end{equation*}
Symmetry of physical laws with respect to time begets conservation of
energy
\begin{equation*}
  E_1 = E_2.
\end{equation*}
Conservation of energy can be written to highlight the work done or the
internal energy change
\begin{align*}
  U_1 + K_1 + W_{\text{other}} &= U_2 + K_2\\
  U_1 + K_1 &= U_2 + K_2 + \Delta U_{\text{int}}.
\end{align*}
The work-energy theorem always relates how much work is required to change the
kinetic energy 
\begin{equation*}
  W = \Delta K.
\end{equation*}

