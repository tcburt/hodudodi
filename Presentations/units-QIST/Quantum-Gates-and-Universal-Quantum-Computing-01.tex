
% ----------------------------------------------------------
\begin{frame}
  \frametitle{Quantum Gates and Universal Quantum Computing}
  $H, S, T,$ and CNOT are a universal set of quantum gates
  \begin{columns}
    % Col 1
    \column[T]{0.45\linewidth}
    %\begin{itemize}
    %\item ``quantum circuits subsume classical circuits''~\cite[\S 4.5]{QCQI-Nielsen}
    %\item 
    %\end{itemize}

    % Phase (S)
    \begin{block}{Phase}
      \begin{quantikz}
        \lstick{\ket{1}} & \gate{S} & \rstick{$e^{i\pi/2}\ket{1}$} \qw
      \end{quantikz}
        \begin{equation*}
          \begin{pmatrix}
            1 & 0 \\
            0 & i
          \end{pmatrix}
        \end{equation*}
    \end{block}

    % $\pi/8$
    \begin{block}{$T$ (also known as $\pi/8$)}
      \begin{quantikz}
        \lstick{\ket{1}} & \gate{T} & \rstick{$e^{i\pi/4}\ket{1}$} \qw
      \end{quantikz}  
        \begin{equation*}
          \begin{pmatrix}
            1 & 0 \\
            0 & e^{i\pi/4}
          \end{pmatrix}
          = %
          e^{i\pi/8}
          \begin{pmatrix}
            e^{-i\pi/8} & 0 \\
            0 & e^{i\pi/8}
          \end{pmatrix}
        \end{equation*}
    \end{block}

    % Col 2
    \column[T]{0.45\linewidth}
    % Hadamard (H)
    \begin{block}{Hadamard}
      \begin{quantikz}
        \lstick{\ket{0}} & \gate{H} & \rstick{$(\ket{0} + \ket{1})/\sqrt{2}$} \qw
      \end{quantikz}   
        \begin{equation*}
          \begin{pmatrix}
            1 & 1 \\
            1 & -1
          \end{pmatrix}
        \end{equation*}
    \end{block}

    % Controlled not 
    \begin{block}{CNOT (two-qubit gate)}
      \begin{tiny}
        \begin{quantikz}
          \lstick{$\ket{c}$} & \ctrl{1} & \rstick{$\ket{c}$} \qw \\ %
          \lstick{$\ket{t}$} & \targ{} & \rstick{$\ket{c}\ket{t\oplus c}$}\qw
        \end{quantikz}
      \end{tiny}
      \begin{Tiny}
        \begin{equation*}
          \begin{pmatrix}
            1 & 0 & 0 & 0 \\
            0 & 1 & 0 & 0 \\
            0 & 0 & 0 & 1 \\
            0 & 0 & 1 & 0 \\
          \end{pmatrix}
        \end{equation*}
      \end{Tiny}
      \vspace*{-\baselineskip}
    \end{block}

  \end{columns}
\end{frame}
